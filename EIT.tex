\documentclass[journal,twocolumn]{IEEEtran}
% documentclass[journal, twocolumn]{IEEEtran}
\usepackage{amsfonts}
\usepackage{amsmath,amssymb}
\usepackage{acronym}  % make an acronym
\usepackage{algorithm}
\usepackage{algorithmic}
% \usepackage{balance}
\usepackage{bm}
\usepackage{bbm}
\usepackage{booktabs}
\usepackage{color, soul}
\usepackage{cite}
% \usepackage{flushend}	% Kunzan leads to problems. 
\usepackage{graphicx}
\usepackage{indentfirst}
\usepackage{lipsum}
\usepackage{setspace}
\usepackage{tikz}
\usetikzlibrary{arrows}
\usepackage{subfigure}
\usepackage[amsmath,thmmarks]{ntheorem}
\usepackage{theorem}
% Enable Hyper-references.
\usepackage{hyperref}
\hypersetup{hidelinks, 
colorlinks=true,
allcolors=black,
pdfstartview=Fit,
breaklinks=true}

\def \T {^{\mathsf{T}}}
\def \H {^{\mathsf{H}}}

\def \opt {^{\text{opt}}}
\def \v {\bm v}
\def \w {\bm w}
\def \g {\bm g}
\def \f {\bm f}
\def \T {\bm \Theta}
\def \t {\bm \theta}
\def \x {\bm \xi}
\def \Pmax {P_{\text{max}}}
\def \ml {multi-layer }
\def \tb {transmit beamformer }
\def \sl {single-layer }
\def \diag {\text{diag}}
\def \opt {^{\text{opt}}}
\def \exp {\text{exp}}
\def \arg {\text{arg}}
\def \CN {\mathcal{CN}}
\def \VM {\mathcal{VM}}
\def \re {\text{Re}}
\def \nc {\mathcal{NC}}
\newcommand{\red}[1]{{\color{red}{#1}}}
\newcommand{\blue}[1]{{\color{blue}{#1}}}

\begin{document}

\title{Electromagnetic Information Theory: Fundamentals, Modeling, Applications, and Open Problems}

% \author{{Jieao~Zhu, Zhongzhichao~Wan, and Linglong~Dai, {\textit{Fellow, IEEE}}
% }
\author{{Jieao~Zhu, {\textit{Student Member, IEEE}}, Zhongzhichao~Wan, {\textit{Student Member, IEEE}}, Linglong~Dai, {\textit{Fellow, IEEE}},\\
M\'{e}rouane Debbah, {\textit{Fellow, IEEE}}, and H. Vincent Poor, {\textit{Life Fellow, IEEE}} }
\thanks{J. Zhu, Z. Wan, and L. Dai are with the Department of Electronic Engineering, Tsinghua University as well as the Beijing National Research Center for Information Science and Technology (BNRist), Beijing 100084, China (e-mails: \{zja21, wzzc20\}@mails.tsinghua.edu.cn, daill@tsinghua.edu.cn).

M. Debbah is with the Technology Innovation Institute, 9639 Masdar City, Abu Dhabi, United Arab Emirates (email:merouane.debbah@tii.ae) and also with CentraleSupelec, University Paris-Saclay, 91192 Gif-sur-Yvette, France. 

H. Vincent Poor is with the Department of Electrical and Computer Engineering, Princeton University, USA (e-mail: poor@princeton.edu).
}}

\maketitle

\begin{abstract}
	\red{Traditional massive multiple-input multiple-output (MIMO) information theory adopt non-physically consistent assumptions, including scalar-quantity, far-field, discretized, and monochromatic EM fields, which mismatch the nature of the underlying electromagnetic (EM) fields supporting the physical layer of wireless communication systems. }
 	Thus, it is of interest to examine the information-carrying capability of continuous EM fields, which motivates research on EM information theory (EIT). In this article, we systematically investigate the basic ideas and main results of EIT. First, we review the fundamental analytical tools of classical information theory and EM theory. Then, we introduce the modeling and analysis methodologies of EIT, including continuous field modeling, degree of freedom, mutual information, and capacity analyses. After that, several EIT-inspired applications are discussed to illustrate how EIT guides the design of practical wireless systems.  Finally, we point out several open problems of EIT, where further research efforts are required for EIT to construct a unified interdisciplinary theory.
\end{abstract}

\begin{IEEEkeywords}
    Electromagnetic information theory (EIT), electromagnetic channel modeling, random field, degrees of freedom (DoF), capacity.  
\end{IEEEkeywords}

\vspace{-1em}
\section{Introduction}
%% leading paragraph. 
% Existing wireless information theory models transceivers as spatially discrete points. For example, single-input single-output (SISO) information theory considers a single point as either a transmitter or a receiver, and multiple-input multiple-output (MIMO) information theory~\cite{goldsmith2003capacity} utilizes discrete points in space to represent transceivers. 
% However, this discretization does not fully capture the spatially continuous electromagnetic (EM) fields~\cite{huang2020holographic}, which are the physical carriers~\cite{migliore2018horse} of information in wireless systems. 
% This mismatch may cause a performance gap between systems designed using discrete models and those designed using continuous EM fields. 
% Therefore, a theory that can model and analyze the real wireless information system with continuous EM fields is of interest, which motivates the research of EM information theory (EIT).
\red{The past decade has witnessed the proliferation of the massive multiple-input multiple-output (MIMO)~\cite{marzetta2010noncooperative} from a theoretical concept to a practical technology. Thanks to the increase of transceiver antennas, the massive MIMO technology has triggered a significant performance improvement in 5G wireless communications. However, prevailing analysis and design procedures for massive MIMO are usually based on scalar-quantity, far-field~\cite{liu2018novel}, discretized~\cite{goldsmith2003capacity}, monochromatic, and other non-physically consistent assumptions, which will lead to mismatches between the system design and the actual propagation environment in the novel architectures beyond massive MIMO. Therefore, a theory that can model and analyze the real-world EM wireless information system with physically interpretable and mathematically reasonable assumptions is of interest, which motivates the research of EM information theory (EIT). } 


EIT is an interdisciplinary subject that integrates deterministic physical theory and statistical mathematical theory to study information transmission mechanisms in spatially continuous EM fields. 
Specifically, EIT unifies the basic laws and methodologies in both classical EM theory and information theory. 
As a result, EIT is capable of building a framework for system modeling and performance analysis that incorporates EM propagation in the analysis of wireless information systems. 

%% Our contributions.
In this article, we systematically investigate the basics and results of EIT. The basic ideas and main results of this article can be summarized as follows:
\begin{itemize}
\item{The fundamental mathematical tools for classical information theory and EM theory are briefly reviewed at first. Specifically, degrees of freedom (DoF), prolate spheroidal wave functions (PSWFs), channel capacity, and Maxwell's equations are introduced. To integrate the probabilistic nature of information theory and the continuous property of EM fields, random fields are then introduced into EIT, which enables performance analysis based on EIT. }
\item{The basic EIT modeling methodologies are introduced for EM channels and EM noise. For EM channel modeling, both the deterministic and stochastic modeling approaches are discussed. For EM noise modeling, the noise fields are first categorized according to their different physical origins, and then the spatial correlation characteristics of different kinds of noise fields are discussed separately. }
\item{EIT performance analysis methods are discussed. By applying the EIT modeling methodologies, the DoF, mutual information, and capacity can be derived using mathematical tools including the PSWFs, Karhunen-Lo\`{e}ve expansion, and random field theories. } 
\item Moreover, several EIT-inspired applications are provided. For example, to approach the EIT capacity, the continuous aperture MIMO (CAP-MIMO) has been proposed by fully harvesting the mutual information in a limited aperture area. Benefiting from the extra spatial DoF predicted by EIT, the location division multiple access (LDMA) technology has been proposed to provide a new possibility for capacity improvement. % In addition, a distance-aware precoding (DAP) scheme has been proposed to take advantage of the extra channel DoF in the near-field region.  % Other applications include precise field extrapolation and reconstruction based on the statistical characteristics provided by EIT modeling. 
\item{Finally, several open problems of EIT are discussed, including the analog and digital challenges brought by EIT, compatible EM noise modeling, and the evaluation of the EM capacity. We explain the meanings of these open problems, and why they are important for future research in EIT. }
\end{itemize}

\vspace{-1em}
\section{Fundamentals of EIT}
In this section, we first distinguish two different definitions of DoF in information theory, i.e., channel DoF and functional DoF.  
Then, we discuss the concept of channel capacity and its relation to DoF. 
After that, we introduce Maxwell's equations to describe the propagation of EM fields. 
Finally, we employ random fields as the EIT foundation for unifying the probabilistic information theory and deterministic field theory. 

\begin{figure}
	\centering 
	\includegraphics[width=\linewidth]{figures/PSWFs.pdf} 
	\caption{\red{Example of the functional DoF: The bandlimited AWGN channel and its eigenmodes. The transmitted signal $x(t)$ is time-limited by the temporal truncation operator $\mathcal{T}_T$ before undergoing the $W$-bandlimited channel and a second truncation $\mathcal{T}_T$. The output of this channel $y(t)$ is then obtained by imposing an AWGN process $n(t)$. The eigenmodes of this linear system are given by the prolate spheroidal wave functions (PSWFs) $\{\psi_n\}_{n=0}^{\infty}$, with eigenvalues $\{\lambda_n\}_{n=0}^{\infty}$. In this figure, $T=2\,{\rm s}$, $W=2\,{\rm Hz}$, and thus the functional DoF of the recieved signal $y(t)$ is $2WT=8$. }}
	\label{fig:PSWF}
\end{figure}

\vspace{-1em}
\subsection{Degrees of Freedom}
\label{Sec_2_Subsec_1}
% introduce the channel DoF. 
DoF is a mathematical quantity that describes the number of independent parameters in a physical system. 
In the communication community, the DoF usually refers to the number of orthogonal subchannels that can independently carry information, which is uniquely determined by the channel structure. It is usually calculated by counting the significant singular values of a given channel matrix \cite{goldsmith2003capacity}. 
Thus, we use the term {{\emph{channel DoF}}} to describe such a number associated with a channel.

% introduce the functional DoF. The formulas 2WT+o(2WT) is indispensable here. 
\red{By contrast, in the information theory community, the DoF has a mathematically rigorous definition that is firmly rooted in the spectral theory in functional analysis. 
Generally, if a normed functional space $\mathcal{X}$ contains an $N$-dimensional subspace $\mathcal{X}_N$ such that any function $f\in\mathcal{X}$ can be ``well-approximated'' by some $\hat{f}\in\mathcal{X}_N$, then it is reasonable to claim that the space $\mathcal{X}$ has an essential DoF of $N$.  
An inspiring and important example would be evaluating the DoF of a waveform channel bandlimited to $[-W, W]\, {\rm Hz}$, as is shown in Fig.~\ref{fig:PSWF}. Equivalently, we want to know how many real numbers can be communicated per unit time through this $W$-bandlimited channel. 
Slepian solved this problem by collecting all the $W$-bandlimited signals together into a functional space $\mathcal{B}_W$, and asking at least how many coefficients $\{x_n\}_{n=1}^{N_\epsilon}$ are needed to approximate an arbitrary $f(t)\in \mathcal{B}_W$ up to a precision of $\epsilon$, i.e., $\|f(t)-\sum_{n=1}^{N_\epsilon}x_n\phi_n(t)\|/\|f\|\leq \epsilon$.
This minimum number of coefficients $N_\epsilon$ is called the Kolmogorov $N$-width $N_\epsilon(\mathcal{B}_W)$ of the functional space $\mathcal{B}_W$ under given norm $\|\cdot\|$. Since data transmission usually takes place within a finite time interval $[-T/2, T/2]$, the norm is chosen to be $L^2(-T/2, T/2)$. Then, the conclusion is that $N_\epsilon=2WT+\mathcal{O}(\log (WT))$~\cite{slepian1961prolate} holds for any given $\epsilon>0$. 
This means that the $W$-bandlimited channel can essentially transmit $2WT$ real numbers within time $T$. 
In this case, the optimal basis waveforms $\psi_n(t)$ are prolate spheroidal wave functions (PSWFs), which are shown in Fig.~\ref{fig:PSWF}. Note that the functional DoF of the received signal $y(t)$ is determined by evaluating the largest $n$ before the eigenvalues $\lambda_n$ exhibit a cut-off transition behavior. 


}

% From a rigorous point of view, the (asymptotic) DoF is defined as the minimum dimension of a finite-dimensional functional subspace that can approximate any function in a larger space with an arbitrarily small error~\cite{poon2005degrees}. For example, the DoF of $W$-bandlimited signals with observation duration $T$ can be derived by finding the finite-dimensional subspace, which contains all the $W$-bandlimited functions that are most concentrated in the time domain. The problem of finding these most concentrated functions is named Slepian's concentration problem~\cite{slepian1976bandwidth}, and it has explicit solutions called prolate spheroidal wave functions, as shown in Fig.~\ref{fig:PSWF}.
% More interestingly, it is proved that to approximately represent any $W$-bandlimited function, it suffices to pick out asymptotically $2WT$ prolate spheroidal wave functions as bases. Thus, the functional DoF of the $W$-bandlimited signals is asymptotically $2WT$~\cite{slepian1976bandwidth}. 
% Since this kind of DoF numerically characterizes the possibility to approximate an {\emph{infinite-dimensional}} functional space by a {\emph{finite-dimensional}} subspace, we refer to it as the {{\emph{functional DoF}}}. 
% It is worth noting that the functional DoF can be alternatively interpreted as the number of minimal required samples to approximately reconstruct an unknown signal. 



% introduce PSWF and its connection to the evaluation of functional DoF. 
% The explicit evaluation of the functional DoF of arbitrary given functional space is hard in general. However, in some special but important cases this task can be fulfilled by leveraging the constraints that define the functional space.
% For example, the functional space $\mathcal{B}_W$ that contains all the $W$-bandlimited functions can be decomposed by finding those particular bandlimited functions that are most concentrated in the time domain. The problem of finding these most concentrated functions, which is named as Slepian's concentration problem, has explicit solutions called PSWFs.
% More interestingly, it is proved that in order to approximately represent any function in $\mathcal{B}_W$, it suffices to pick out $2WT+o(2WT)$ PSWFs as basis functions. Thus, the functional DoF of $\mathcal{B}_W$ is $2WT+o(2WT)$. 

% Similar to the singular decomposition method for evaluating the channel DoF, the functional DoF of a given signal space is evaluated by approximating the signal space by a finite-dimensional subspace. In the case of band-limited functional space, this decomposition where the basis functions of the subspace are obtained by solving a the Slepian's concentration problem, and the solutions are proved to be the prolate spheroidal wave functions (PSWFs). 

\vspace{-1em}
\subsection{Channel Capacity}
\label{Sec_2_Subsec_2}
Different from the notion of channel DoF, which characterizes the number of orthogonal subchannels available, the channel capacity measures the error-free information transmission capability of the channel. 
Since the transmission errors are caused by random channel noise, the channel is information-theoretically defined to be a conditional transition probability that randomly maps the input to the output of the channel. 
Then, the mutual information between the input and output can be defined, and the channel capacity is derived by taking the supremum of the mutual information over all possible input distributions of the channel. 

The operational meaning of the channel capacity is established by Shannon in his seminal paper~\cite{shannon1948mathematical} in 1948. To establish the operational meaning, he proved the {\it achievability} and {\it converse} theorem of the channel capacity. The {\it achievability} states that, for any given error probability and data rate lower than the capacity, there exists a pair of encoder-decoder of sufficient code length to operate below such an error probability. 
The {\it converse} theorem states that, for an arbitrary data rate higher than capacity, no matter what kind of code is employed, the error probability is bounded away from zero. This very important result indicates that, it is impossible to realize error-free transmission at a rate higher than the capacity. Thus, the value of channel capacity is established to be the fundamental transmission limit of a given channel. 

\begin{figure}
	\centering 
	\includegraphics[width=0.9\linewidth]{figures/LTI-LSI-new-v1.pdf} 
	\caption{\red{In analogy with linear time-invariant systems described by a time-domain impulse response $h(t)$ in classical information theory, the EIT model is based on linear space-invariant systems described by the Green's function ${\bm G}({\bm r}, {\bm s})$.} } 
	\label{fig:LTI_LSI}
\end{figure}
% Original figure: figures/LTI_LSI.pdf 

\vspace{-1em}
\subsection{Electromagnetic Theory}
\label{Sec_2_Subsec_3}
% Basics about electromagnetics.
EM theory is a branch of physics that studies the EM interaction in four-dimensional spacetime from a field-theoretic point of view. The EM forces are carried by EM vector fields, which are usually described by Maxwell's equations. Maxwell's equations are four linear partial differential equations that characterize how the electric fields and magnetic fields are altered by each other and by charges and currents. Combining these four equations, a vector wave equation can be obtained that manifests the existence of EM waves. 

\red{In wireless communications, to describe the EM response ${\bm E}({\bm r})$ at the receiver induced by the source current ${\bm J}({\bm s})$ at the transmitter, Green's function ${\bm G}({\bm r}, {\bm s})$ is usually introduced as the spatial impulse response of EM systems~\cite{stratton2007electromagnetic}, as is shown in Fig.~\ref{fig:LTI_LSI}.
Since the electric and magnetic field vectors are all three-dimensional time-varying vectors, to describe the mapping between two vectors, Green's function must take the form of a dyadic-valued function. 
Note that EM theory is a deterministic theory built on partial differential equations and the knowledge of all surroundings, so Green's function is uniquely determined by the boundary conditions of EM problems.
As a result, pure EM theory cannot capture the random nature that arises in wireless communications. }

\subsection{Random Fields for EIT}
\label{Sec_2_Subsec_4}
\red{In order to apply information-theoretic analysis to EM fields, probabilistic measures should be assigned to the functional space containing all possible EM fields that satisfy Maxwell's equations, which leads to the random field modeling for EIT. 
Gaussian random processes, usually indexed by the continuous time $t$, are widely applied to model communication signals, from which entropy rates and mutual information formulas can be derived. Gaussian random fields (GRFs) are generalized random processes that allow multiple index variables. 

Mathematically, GRFs are random multivariate functions whose finite-dimensional marginal distributions are Gaussian. Justified by the central limit theorem, many complicated spatially dependent random values that arise in wireless communications can be modeled by GRFs.  
Inheriting favorable mathematical properties from Gaussian distributions, GRF models exhibit good analytical properties that can facilitate both theoretical deduction and numerical inference. 

In EIT, GRFs can be utilized to model both the EM channels~\cite{marzetta2022fourier} and the EM signals~\cite{wan2022mutual}. 
In real-world communication systems, the EM channel response may vary randomly as a function of spacetime, so it is reasonable to model such spatio-temporal variations by a channel GRF. 
This allows the Bayesian inference of channel entries that are not directly measured. 
Similarly, in order to capture the randomness of the transmitted and received signals, the current distributions at the transmitter and the electric fields at the receiver can also be modeled by GRFs~\cite{wan2022mutual}, from which the information-theoretic mutual information can be defined to characterize the information transmission capability of such EM channels. The GRF-based information-theoretic analysis will be explained in detain in Section.~\ref{sec_3_subsec_3}. }

% Motivated by the possibility of exploiting the three spatial dimensions, the multi-input multi-output (MIMO) architecture have been ubiquitous in contemporary wireless communication systems. To fully exploit the potential of MIMO, numerous spatial multiplexing technologies have been proposed, including SDMA, TRDMA, PDMA, and LDMA. However, there must be a unified underlying theory that governs all the multiplexing technologies, since they are all based on the same physical media of electromagnetic waves. To construct this unified theory, two different approaches have been proposed in the literature to characterize the electromagnetic channel. 

% The first approach is to model the electromagnetic channel in a deterministic manner. The determination of the modeling means the channel is treated as a fully predictable function of spacetime, as long as the geometric parameters of the transceivers and the scatterers are known. A typical modeling of such a deterministic channel is describing the electromagnetic channel with the dyadic Green's function. The dyadic Green's function, which can be directly derived by solving the vector wave equation of electromagnetic fields, can be regarded as the spatial impulse response of the electromagnetic field with given boundary conditions. Similar to the decomposition of a $W$-bandlimited time-domain waveform channel, the four-dimensional electromagnetic channel, described by some deterministic Green's function, can be decomposed onto a certain orthonormal eigenfunctions, and the corresponding eigenvalue set determines all the information-theoretic numerical characteristics of such channel, including the DoF and capacity.

% The second approach is to model the electromagnetic channel in a stochastic manner. Instead of treating the electromagnetic channel by some deterministic impulse response, this stochastic modeling regards the channel itself as a random field, in which the electromagnetic constraints brought by Maxwell's equations are encoded in the auto-correlation function of such a random field. By this approach, we can obtain the ergodic capacity, but the exact DoF cannot be defined with such a stochastic assumption.  


% In wireless communications, electromagnetic waves are often exploited to carry information by manipulating their amplitudes, frequencies, and phases. To transmit and receive the electromagnetic waves, antennas are designed to radiate the waves into the free space at the transmitter, and couple the waves into electronic circuits at the receiver. Thus, from an end-to-end point of view, by applying antennas, the electromagnetic channel in four-dimensional spacetime is simplified into a time-domain waveform channel with only one temporal dimension. 
% In this simplified model, both the electromagnetic properties of the propagation environment and the detailed design of the antennas are condensed into an end-to-end impulse response $h(t)$ of the time-domain equivalent channel, to which Shannon's information-theoretic approach in the previous Subsection~\ref{Sec_2_Subsec_1} can then be applied. 
% It is worth noting that, with this end-to-end single-input single-output (SISO) model, only one time dimension out of four spacetime dimensions of the electromagnetic waves can be efficiently utilized. Thus, the DoF of the electromagnetic channel is well under-exploited in SISO technology. 


\section{Modeling Methodologies for EIT}
In this section, we discuss the continuous modeling schemes for EIT, which include both channel modeling and noise field modeling. 

\vspace{-1em}
\subsection{Continuous Channel Modeling}
\red{In this part, we will discuss the channel models in EIT. The basic idea is to describe the EM channel by a bounded linear operator $T$ that maps the source current distribution ${\bm J}({\bm s})\in\mathcal{L}^2(V_{\rm T})$ to the noiseless received field ${\bm E}({\bm r})\in\mathcal{L}^2(V_{\rm R})$.   
% Since spatial multiplexing can only be realized with MIMO technology, we will always assume a MIMO transceiver in the sequel. 
Note that to model a traditional narrowband MIMO channel, a matrix with complex-valued entries ${\bm H}:\mathbb{C}^M\to\mathbb{C}^N$ is usually utilized, where each entry represents the complex channel between a pair of transceiver antennas. 
This channel model is, in essence, spatially discrete. 
However, in EIT, it is reasonably assumed that one can measure the field at an arbitrary point inside the receiver region, where the precision of such measurement is subject to some physical noise limits. This leads to the assumption of EIT that the transceivers operate in a continuous functional space $\mathcal{L}^2(V_{\{{\rm T, R}\}})$.  }

\red{To ensure compatibility with discrete MIMO channel modeling, the connection between the entries of ${\bm H}$ and the transceiver antenna modes are usually assumed to be ${\bm H}_{qp}=\langle\varphi_q|T|\phi_p \rangle$, where $\phi_p$ and $\varphi_q$ are the $p$-th and $q$-th operating modes of the transmit and receive antennas, respectively. Thus, modeling the EM channel by a bounded linear operator $T$ is mathematically consistent with the existing matrix modeling. }
% Mathematically, an antenna acts as a continuous integral operator that weights the incident signal by its EM eigenmode, which corresponds with the continuous modeling in EIT. 
% Thus, the EIT channel is expected to be modelled by a continuous linear operator, where the operator can be either deterministic or stochastic.

\red{The EIT channel can either be deterministically modeled or stochastically modeled. Deterministic channel models have simpler mathematical expressions, but stochastic models usually better capture the small-scale fast fading caused by multipath effects and user mobility. }
The simplest deterministic channel model is the line-of-sight (LoS) model, which assumes that there is only one direct link without any scatterers between the transceivers. In this scenario, the channel operator is the free-space Green's function ${\bm G}({\bm r}, {\bm s})$. 

\begin{figure}
	\centering 
	\includegraphics[width=\linewidth]{figures/random_channel-new.png} 
	\caption{Continuous stochastic channel modeling, where the linear channel operator is projected onto the Fourier plane-wave basis~\cite{marzetta2022fourier}.} 
	\label{fig:marzetta}
\end{figure}
\red{Another approach is stochastic channel modeling, which usually specifies the autocorrelation function ${\bm R}({\bm r}, {\bm s}; {\bm r}', {\bm s'})$ between the ${\bm s}\to {\bm r}$ channel and the ${\bm s}'\to{\bm r}'$ channel coefficients. As introduced in Subsection~\ref{Sec_2_Subsec_4}, this autocorrelation function determines a GRF that enables useful tools such as Bayesian inference for channel estimation. 
For stochastic channel modeling, a recent novel method is named as Fourier plane-wave expansion (see Fig.~\ref{fig:marzetta}), where the channel is modeled by a special GRF which is provably compatible with the EM Helmholtz equation~\cite{marzetta2022fourier}. This EM-compatible modeling enables further information-theoretic analysis that reveals the fundamental limit of electromagnetic laws on information transmission. }

\vspace{-3mm}
\subsection{Noise Field Modeling}
After the discussion of the continuous random channel models, we will introduce the noise modeling scheme. 
Noise is vital in information theory, because it is a key factor to determining the capacity. 
\red{ In classical information theory, the noise is usually modeled by a time-domain additive white Gaussian noise (AWGN) with a constant power spectral density $n_0/2$. Thus, its projection coefficients onto any orthonormal basis are independent and of equal power $n_0/2$, which simplifies theoretical analysis. 
Similarly, in EIT, the noise is usually modeled as a spatial AWGN with flat wavenumber PSD. This spatial AWGN implies that the additive noises at any disjoint small spatial regions $V_1, V_2$ are independent and identically distributed (i.i.d.) complex Gaussian random variables, whose variances are proportional to the volumes $\mu(V_1), \mu(V_2)$ of the small regions. Although the AWGN model facilitates theoretical analysis, the white spectral assumption is problematic, since it causes an unbounded noise power. }

% The white noise PSD is caused by the uncorrelated assumption in the spatial domain. In general, the noises at different points of space cannot be assumed independent, i.e., they do exhibit some spatial correlation. 
% The reason is that, the noise is not completely composed of spatially uncorrelated thermal noise. 
% Instead, the unwanted interference waves wandering in the space also enter the receiver, leading to an equivalent noise which is spatially correlated in general. 
% To characterize this EM interference, the one-ring model~\cite{byers2004spatially} is borrowed from prior works on channel modeling and is extended to a more general one-sphere scenario with polarized EM waves~\cite{wan2022mutual}, where the analytical expressions of the spatial correlation function can be obtained. 
% This non-white spatially correlated noise model is of theoretical importance, since it is provably compatible with Maxwell's equations. 
% By contrast, other noise models including the sinc-shaped or jinc-shaped correlation functions~\cite{marzetta2022fourier} fail to satisfy Maxwell's constraints because they ignore the polarization of the EM waves. 

\section{Information-Theoretic Analysis for EIT}
In this section, based on the EIT modeling methodologies above, we discuss the basic performance indicators and the corresponding analysis techniques of EIT, including the functional DoF, the channel DoF, the mutual information, and the capacity. 
% For a wireless system, the DoF shows the maximum number of independent channels that can be utilized to transmit information. For example, the DoF of a traditional MIMO system is usually approximated by the minimum value of the number of transmitting antennas and the number of receiving antennas. However, such approximation fails when we consider massive antennas in a spatially-constraint region, because the approximated DoF will tend to infinity with the growth of the number of antennas. Therefore, the true DoF value of a spatially constrained continuous wireless system is worth analyzing. In this subsection, we will discuss the functional DoF and channel DoF in EIT separately, and summarize the related works.

\vspace{-1em}
\subsection{Functional DoF}\label{Sec_4_Subsec_1}
As we have clarified in Section~\ref{Sec_2_Subsec_2}, the functional DoF in classical information theory represents the minimum number of required samples to reconstruct the signal. Similarly, in EIT, the functional DoF refers to the minimum number of required samples $N_\epsilon$ to reconstruct a given EM field. 
Such a functional DoF is closely related to the transmission capability of the EM system, because the minimum number of required samples to reconstruct an EM field is equivalent to the maximum number of complex values that can be transmitted within a single EM channel use. 

\red{The functional DoF of a scattered EM field ${\bm E}({\bm r})$ can be analyzed in the transform domain ${\bm E}({\bm k}) = \mathcal{F}^3[{\bm E}({\bm r})]({\bm k})$, i.e., the wavenumber domain. Resembling the bandlimited signals in the time-frequency domain, the radiated EM fields also exhibit the wavenumber-limited property in the space-wavenumber domain. 
Thus, it can be easily proved that a half-wavelength sampling suffices to asymptotically reconstruct an arbitrary EM field up to any given precision $\epsilon$ as the receiving region $\mu(V_{\rm R})\to\infty$, i.e., the EM DoF is at most proportional to the number of half-wavelength grids in the receiver region. }
% If we observe the EM field in a spatially limited region, of which the size is far larger than the wavelength of the EM field, then we can perform half-wavelength sampling on the region. The DoF obtained from the region is asymptotically proportional to its length or area. 

A more refined analysis of the wavenumber-limitedness of the EM fields shows that the EM functional DoF depends on how rapidly the phase of a radiated field changes over space. 
\red{To describe such a phase change, the authors of~\cite{bucci1987spatial} introduced the spatial bandwidth $W$ to describe the degree of wavenumber-limitedness. 
It is proved that for electromagnetic sources confined to a sphere of radius $a$, the spatial bandwidth satisfies $W\leq \sqrt{2}\beta a$, where $\beta$ is the propagation constant of the time-harmonic EM fields. 
The functional DoF of the received field is thus proportional to $W$ and the length of the observation region.  }

% In the analysis approach of spatial functional DoF, a notion named spatial bandwidth is introduced, which extends the bandwidth in classical information theory to spatial domain~\cite{bucci1987spatial}. The bandwidth in classical information theory depicts the band-limited characteristics of the signal in the frequency domain, and can be used to derive the DoF of the signal according to the Nyquist sampling theorem. Similarly, the spatial bandwidth depicts the band-limited characteristics of the EM field in the wavenumber domain, and can be used to derive the DoF of the EM field in the space-wavenumber domain. For a given channel model, the spatial bandwidth can be approximated using the method of steepest descent. More general results considering the four-dimensional spacetime are based on Landau's eigenvalue problem.
% \begin{figure*}[ht]
% 	\centering 
% 	\includegraphics[width=0.81\linewidth]{figures/CAPMIMO.pdf} 
% 	\caption{CAP-MIMO transceivers~\cite{zhang2022pdma}.  }
% 	\label{fig:CAPMIMO}
% \end{figure*}


\vspace{-1em}
\subsection{Channel DoF}
The functional DoF introduced in the previous subsection emphasizes the intrinsic DoF of the received field. 
However, due to the restrictions of the EM channel, it is possible that some of these functional DoFs at the receiver cannot be excited. For example, an EM propagation environment with rich scatterers can excite almost all angular eigenmodes at the receiver, while an EM channel only composed of LoS paths may not enable all eigenmodes. Thus, evaluating the channel DoF is of practical importance.  

\red{As is defined in Subsection~\ref{Sec_2_Subsec_1}, the channel DoF represents the maximum number of independent parallel channels that can be used to transmit information. 
Similar to the SVD decomposition of matrices, the DoF of a LoS EM channel can be solved by expanding the channel operator $T$ onto a series of orthogonal sub-channels: $T=\sum_n \sigma_n |\varphi_n\rangle\langle\phi_n|$, and counting the number of channel gains $\sigma_n$ that exceed a certain threshold $\epsilon$. 
Specifically, in the scenario where a pair of coaxial square transceivers are employed as transmit and receive antennas, the LoS channel operator $T$ is given by the free-space Green's function ${\bm G}({\bm r}, {\bm s})$, and the corresponding eigenvalue problem can be approximated by the standard Slepian's concentration problem~\cite{miller2000communicating}. In this case, the channel eigenmodes $\varphi_n$ and $\phi_n$ are proved to be well-approximated by prolate spheroidal wave functions $\psi_n$.   
Through this approach, the DoF of LoS channel model has been approximately derived to be proportional to the product of the area of the transceivers~\cite{pizzo2022nyquist,miller2000communicating}. }

For non-line-of-sight (NLoS) deterministic channel models, the channel DoF can be similarly derived by performing Slepian's analysis in the angular domain, since the scatterers usually appear in limited solid angular regions. 
Following Slepian's analysis, it is proved that the NLoS channel DoF is proportional to both the solid angle of the scatterer cluster and the geometric lengths of the transceivers~\cite{poon2005degrees}.   
If the channel is modeled as a stochastic channel with random scatterer positions, von Mises-Fisher distributions~\cite{byers2004spatially} can be used to model the statistical characteristics of the channel. 
The channel DoF in this stochastic model should be the expected value averaged by probability.

\begin{figure}
	\centering
	\includegraphics[width=1.0\linewidth]{figures/discrete_receiver.pdf}
	\caption{Convergence of the discrete MIMO mutual information to the continuous-space EIT mutual information $I({\bm J}; {\bm E})$. The $x$-axis (sampling number) represents the number of antennas placed in the receiver region $V_{\rm R}$. }
	\label{fig:MIMO_EIT_comparison}
\end{figure}

\vspace{-3mm}
\subsection{Mutual Information and Capacity Analysis} \label{sec_3_subsec_3}
\red{Besides the DoF which reveals the number of available sub-channels, the channel capacity $I({\bm J}; {\bm E})$ is another important performance indicator showing the maximum information transmission rate of a wireless system. 
Compared to the traditional MIMO information theory using random vectors as transmitted and received signals, EIT treats the EM fields ${\bm J}, {\bm E}$ as random fields. 
The random field modeling follows the statistical approach of mutual information and capacity analysis by Shannon. 
Each realization of the random field represents a possible pattern of the radiating field. The ensemble of the realizations, to which a probability measure is assigned, depicts the statistical characteristics of the system. }

The mutual information and capacity analysis based on random fields is a continuous extension of the analysis based on random vectors. 
\red{In parallel with MIMO theory based on matrices, operator theory can be used to describe the autocorrelation ${\bm R}({\bm r}, {\bm r}')=\mathbb{E}[{{\bm E}({\bm r}){\bm E}\H ({\bm r}')}]$ of the random field. 
According to Mercer's theorem, the continuous channel can be decomposed into parallel sub-channels. 
Thus, the information obtained from the received random field is evaluated by summing the information over all sub-channels. 
Different from the matrix determinant form of MIMO mutual information, the EIT information takes a Fredholm determinant form $\log\det({\bf I}+T_{\bm E}T_{\bm N}^{-1})$~\cite{wan2022mutual}, where $T_{\bm E}$ and $T_{\bm N}$ are self-adjoint autocorrelation operators of the random fields ${\bm E}({\bm r})$, and ${\bm N}({\bm r})$, respectively.  
This Fredholm determinant form provides a closed-form formula for the EIT mutual information. Thus, numerical schemes for evaluating Fredholm determinants can be transplanted onto the computation of this EIT mutual information, as is shown in Fig.~\ref{fig:MIMO_EIT_comparison}. It is numerically verified that the discrete MIMO mutual information converges to the continuous-space EIT mutual information, which justifies the Fredholm definition of the EIT mutual information. }

\section{Applications of EIT}
In this section, we present several EIT-inspired applications. 
These applications either mimic the analysis methodologies of EIT to fully achieve the existing DoF, or try to explore new communication resources predicted by EIT. 

\vspace{-1em}
\subsection{Continuous Aperture MIMO (CAP-MIMO)}
Traditional wireless communication systems deploy finite antennas in a limited aperture at the transceivers. 
However, the EIT mutual information analysis is built on spatially continuous theoretical frameworks. 
To bridge the gap between the performance limit in MIMO theory and that in EIT, CAP-MIMO, which is also called holographic MIMO, has attracted increasing research interests recently~\cite{zhang2022pdma}. 
CAP-MIMO adopts a hypothetical structure that contains infinitely dense antennas in a limited spatial region, which is capable of continuously generating arbitrary current distributions at the transmitter and detecting arbitrary electric fields at the receiver.  
The current distributions on the transmitter are called patterns of the CAP-MIMO. 
Different signals are modulated onto different patterns before radiating into the space. 
These continuous patterns need to be optimized by specially-designed algorithms in order to achieve a higher multi-user data rate~\cite{zhang2022pdma}. 


% Several works have already explored the pattern design scheme for CAP-MIMO system. The simplest and the easiest way to implement is utilizing the Fourier basis as the patterns, which transfer the approach of OFDM scheme in the time-frequency domain to spatial-wavenumber domain. Other special functions are also utilized to design the patterns in the corresponding assumptions like line-of sight communication or single receiver. For general scenarios, the continuous current patterns can be projected to orthogonal basis and optimization schemes can be utilized to obtain the near-optimal solution of the pattern design scheme.

\subsection{Location Division Multiple Access (LDMA)}
\begin{figure*}[t]
	\centering 
	\includegraphics[width=0.9\textwidth]{figures/LDMA.pdf} 
	\caption{Comparison between far-field SDMA and near-field LDMA, where the DoF in the distance domain can be exploited to provide a new possibility for capacity improvement~\cite{wu2022multiple}. }
	\label{fig:LDMA}
\end{figure*}

\red{
	Traditional far-field spatial division multiple access (SDMA) scheme exploits the angular orthogonality of the far-field planar-wave propagation channel to serve users from different angles simultaneously. However, in order to further improve the communication rate, extremely large antenna arrays (ELAAs) are introduced for improving the spectral efficiency. Different from the traditional far-field propagation environment, this enlarged array aperture will inevitably introduce the {\it near-field} spherical wave propagation characteristics.  

	Although the near-field effect seems to corrupt the planar wave assumptions and make the traditional DFT codebook-based angular domain beamforming techniques no longer applicable, it also brings new potential for data transmission and multiplexing. This is because the more complicated near-field propagation channel exhibits additional orthogonality apart from the already well-known angular orthogonality in SDMA-based systems. 
	The additional orthogonality comes from the distance domain, i.e., the channels of users located at different distances are proved to be orthogonal in the near-field region~\cite{wu2022multiple}, thus providing extra DoFs in the distance domain. Thus, by re-designing the near-field codebook to match the near-field propagation characteristics of the EM waves, a new multiple access scheme called the location division multiple access (LDMA) is proposed to enhance the overall multiple access capability of the communication systems. 
}

% For the classical MIMO-enabled multiple access scheme called space division multiple access (SDMA), the beams emitted from the base station form a functional space. 
% The DoF of the functional space represents the resolution of distinguishing users at different locations, which corresponds to the functional DoF analysis in EIT. 

% However, the classical SDMA based on far-field models can only distinguish the users in the angular dimension, but it fails to utilize the functional DoF in the distance dimension. 
% For future wireless communications with extremely large-scale antenna arrays (ELAAs), the more accurate spherical wave model should be used instead of the simple planar wave assumption. 
% With the spherical wave model, in addition to the angular dimension, the users can also be distinguished in the distance dimension. 
% Therefore, a new multiple access scheme called LDMA~\cite{wu2022multiple} was recently proposed to fully exploit the extra functional DoF of the beams so as to simultaneously serve users at different angles and distances, as is shown in Fig.~\ref{fig:LDMA}. 
% In this way, unlike the almost only dominant way of spectrum efficiency enhancement by significantly increasing the number of antennas in 4G and 5G, LDMA provides a new possibility for multi-user capacity improvement by exploiting additional structure of the EM fields for future 6G.   

% In the LDMA scheme, the near-field codebook, beam training, and channel estimation are designed by considering the extra DoF in the distance domain. 

% \vspace{-1mm}
% \subsection{Distance-Aware Precoding}
% As discussed in Subsection~\ref{Sec_4_Subsec_1}, the channel DoF of an LoS channel model in EIT is approximately proportional to the length of the transceivers and inversely proportional to the distance between the transceivers. 
% Accordingly, only one data stream can be supported in the far-field LoS channel, since the channel DoF is only one in this case. 
% However, in the near-field region, the LoS channel has much higher channel DoFs, so more data streams can be simultaneously supported.
% To fully utilize the near-field channel DoF, a distance-aware precoding scheme has been recently proposed, which adaptively adjusts the number of radio frequency chains to match the channel DoFs at different distances for single-user capacity improvement.   

% \subsection{Field Reconstruction}
% In engineering practice, it is often required to reconstruct the values of an EM field from a finite number of available observations, which is called the field reconstruction problem. 
% To solve this problem, prior information about the EM field is often exploited, including the spatially bandlimited property and the spatial correlation that the EM field possesses. 
% These prior information can be provided by an EIT analysis on the field region. 
% For example, the spatial bandwidth property justifies the cardinal series as asymptotically optimal linear interpolators~\cite{pizzo2022nyquist} of the EM field, while the field correlation methodology leads to Gaussian process regression (GPR)-based field interpolators. 
% These field interpolators are widely applicable to microwave field measurement tasks and channel prediction problems. 


\section{Open Problems for EIT}
In this section, we will discuss some open problems for EIT. We will also explain the meanings of these open problems, and why they are important for the future research of EIT. 

\subsection{Digital and Analog Challenges}
To achieve the capacity improvement predicted by EIT, transceivers are expected to employ fully-continuous antenna surfaces. However, the implementation of such spatially continuous antenna surfaces will inevitably require the processing of an enormous volume of baseband data and the deployment of a vast amount of electrically small antenna patches. Thus, the exploding number of antennas will bring fundamentally new challenges to both the digital and analog signal processing. 

In the digital domain, the optimal current distribution patterns should be frequently updated within a typical timescale of 1\,ms to accommodate the fast-varying EIT channel. In addition, the DFT bases have to be further improved since the EIT eigenmodes are generally not sinusoidal~\cite{marzetta2022fourier}. Thus, numerous nearly-continuous bases have to be frequently designed in the digital domain, which is a great challenge even to the state-of-the-art baseband processors.  

In the analog domain, there exists a physical tradeoff between the performance and the size of antennas~\cite{stratton2007electromagnetic}. Thus, it is practically difficult to design near-continuous antennas without sacrificing the radiation efficiency and working bandwidth. Apart from the physical tradeoff, the mutual coupling between adjacent antennas need to be carefully suppressed when the antenna array becomes denser, in order not to churn the designed radiation patterns and degrade the communication performance. 


\subsection{Compatible Noise Modeling in Discrete and Continuous Communication Systems}
It can be proved that, the purely i.i.d. thermal noise distribution will cause the divergence of the end-to-end capacity of a MIMO transceiver, when the number of receiving antennas increases indefinitely within a constrained aperture size. 
The reason is that, when the number of receiving antennas increases, the signals can be aggregated coherently within any small spatial region, while the noises add up non-coherently because they are uncorrelated. 
Thus, in this small region, the signal energy scales quadratically, while the noise energy scales linearly, resulting in an unbounded linear improvement of the signal-to-noise ratio (SNR). Thus, the capacity will diverge to infinity, contradicting the principle of energy conservation.  

This absurdity is, in fact, caused by the improper assumption that the noise is spatially uncorrelated. 
If we assume a correlated noise, then this capacity divergence naturally disappears. 
As a result, correlated noise models are required for the EIT capacity analysis. On one hand, the noise model may be tuned to exhibit a small enough spatial ``coherence length'' to be compatible with the independent MIMO noise model with the half wavelength-spaced antennas. On the other hand, the noise should possess some spatial correlation on a small scale to ensure a finite-valued EIT capacity. 
The construction of such a kind of noise model that bridges the macro-scale and micro-scale noises is, up to date, an open problem. 

% \vspace{-1em}
% \subsection{Power Constraint of Continuous Communication System}
% For both the classical information theory and EIT, the power constraint is a key factor that affects mutual information and capacity. 
% Therefore, imposing appropriate power constraints on the system is of both theoretical and practical significance. 
% In the classical information theory, the power constraint is imposed on the expectation of the squared norm of the signal vector.  
% Similarly, for EIT, usually the functional $\mathcal{L}_2$-norm of the source current is restricted to some pre-determined power values.  

% However, this form of power constraint causes incompatibility between MIMO information theory and EM theory.  
% In MIMO information theory, the signal vectors are complex-valued mathematical abstractions of the actual physical process, so there is no guarantee that the squared norm of such vectors represents the actually radiated power. 
% However, in EM theory, a more reasonable power constraint is given by calculating the total radiated power from an EM theoretical perspective, rather than calculating the $\mathcal{L}_2$-norm of the source current. 
% Unfortunately, these two definitions of power constraints are incompatible and often yield different power values, which is an open problem for EIT. 


\subsection{Evaluation of the EM Capacity}
\red{In classical information theory, the ``capacity'' is defined as the supremum of all the operationally achievable transmission rates. It is favorable that in discrete memoryless channels, the capacity equals the maximum mutual information. However, this conclusion has not been proved for continuous waveform channels. In the existing works on EIT~\cite{wan2022mutual,zhang2022pdma,marzetta2022fourier}, maximum EIT mutual information values are calculated under the assumptions of linear deterministic EM channels and continuous transceivers, and the EIT mutual information is further compared to the discrete MIMO mutual information. Unfortunately, these EIT mutual information values only serve as an upper bound to the EM operational capacity. 
Thus, in the strict sense, the EM capacity is still an open problem. }

\section{Conclusions}
In this paper, we have investigated the fundamental mathematical tools, basic modeling methodologies, and theoretical performance analyses that constitute EIT. 
Furthermore, we have discussed some recent novel applications related to EIT, aiming at designing new wireless communication systems for single-user and multi-user capacity enhancement. 
The recent progress on EIT has demonstrated its potential to become a unified and widely applicable theory. However, there are still some unresolved open problems that require further study in the future. 
%including the unknown compatible EIT noise modeling, contradictive power constraints, the immature proofs of information-theoretic achievability and converse bounds in EIT, and the undiscovered granularity limit of EIT current manipulation. % insights added here. 
% To address these problems, borrowing from the common knowledge in theoretical physics, a special relativity-based EM theory that exhibits more symmetries may be introduced into the study of EIT to facilitate theoretic analysis.  


\footnotesize

\bibliographystyle{IEEEtran}
\bibliography{EIT, IEEEabrv}

\normalsize
\vspace{-1em}
\section*{Biographies}

{\bf Jieao Zhu} is a Ph.D. student from the Department of Electronic Engineering at Tsinghua University, Beijing, China. His research interests include electromagnetic information theory (EIT), coding theory, and quantum computing. 
\\

{\bf Zhongzhichao Wan} is a Ph.D. student from the Department of Electronic Engineering at Tsinghua University, Beijing, China. His research interests include electromagnetic information theory (EIT), coding theory, and channel modeling. 
\\


{\bf Linglong Dai} is an Associate Professor from Tsinghua University. His current research interests include massive MIMO, RIS, Wireless AI, and EIT. He has received the IEEE ComSoc Leonard G. Abraham Prize in 2020, the IEEE ComSoc Stephen O. Rice Prize in 2022, and the IEEE ICC 2022 Outstanding Demo Award. He was listed as a Highly Cited Researcher by Clarivate in 2020 and 2021. He was elevated as an IEEE Fellow in 2022.
\\

{\bf M\'{e}rouane Debbah} is chief research officer at the Technology Innovation Institute in Abu Dhabi. From 2014 to 2021, he was vice-president of the Huawei France Research Center, where he was jointly the director of the Mathematical and Algorithmic Sciences Lab as well as the director of the Lagrange Mathematical and Computing Research Center. 
\\

{\bf H. Vincent Poor} is the Michael Henry Strater University Professor of Electrical Engineering. His research interests are in the areas of information theory and signal processing, and their applications in wireless networks, energy systems and related fields. Dr. Poor is a member of the National Academy of Engineering and the National Academy of Sciences, and is a foreign member of the Chinese Academy of Sciences, the Royal Society and other national and international academies. 
\\

\end{document}


